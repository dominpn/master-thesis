
\chapter{Podsumowanie}

Cele postawione przed pisaniem pracy udało się zrealizować. Przedstawiono najczęściej stosowane algorytmy stosowane podczas tworzenia systemów rekomendacyjnych, główne założenia, problemy i wyzwania, oraz sposoby ewaluacji zaprojektowanych rozwiązań. Pierwsza część pracy wskazuje, że wykorzystanie poszczególnych metod zależy nie tylko od danych jakie gromadzone są przez system, ale również od liczby użytkowników czy specyfikacji (założeń) danego systemu. Połączenie poszczególnych rozwiązań w jedno umożliwia na pozbycie się występujących problemów.

Dalsza część pracy przedstawia problem bezpieczeństwa danych w szeroko pojętych systemach rekomendacyjnych oraz sposobów zwiększających ochronę danych użytkowników podczas procesu przetwarzania danych. Zaproponowany protokół bezpiecznej rekomendacji wykorzystuje opisane w pracy metody (szyfrowanie homomorficzne oraz \textit{Federated learning}) mając na uwadze bezpieczeństwo zarówno klientów jak i samej aplikacji. Rozwiązanie miało na celu zniwelowanie wad metod działających oddzielnie poprzez ich połączenie - takich jak ograniczone zasoby sprzętowe i programowe czy wydłużenie procesu rekomendacji.

Dalsze eksperymenty miały na celu udowodnienie, że wykorzystanie stosunkowo nowego podejścia w uczeniu nie wpływa na pogorszenie skuteczności rekomendacji lub czasu wymaganego do przetworzenia otrzymanych danych. W przypadku wymaganego czasu nie ma to dużego wpływu na działanie systemu, ponieważ sam proces uczenia odbywa się na urządzeniu użytkownika bez ujawniania żadnych danych. Same opóźnienia nie miałyby dużego oddziaływania na działanie całego systemu. Jednak duży wpływ na funkcjonowanie aplikacji ma skuteczność uzyskana przez klientów. Przeprowadzone eksperymenty pokazują, że zastosowanie metody \textit{Federated learning} nie powoduje znacznego spadku otrzymanych rekomendacji. 

Jednak w trakcie implementacji okazało się, że testowane biblioteki nie są jeszcze w takim stanie, aby były wykorzystywane w produkcyjnie działających aplikacji. Spowodowane jest to małą liczbą przykładów zawartych w dokumentacji, stosunkowo małą społecznością użytkowników dzielących się wiedzą oraz wczesną wersją samych bibliotek (w momencie pisania pracy jest to odpowiednio wersja 0.16.1 dla \textit{Tensorflow federated} oraz wersja 0.2.8 dla biblioteki \textit{PySyft}). Wczesne wersje mogą być niestabilne, a także w trakcie rozwoju może dojść do znaczących zmian powodujących brak kompatybilności wstecznej.

Mimo licznych problemów napotkanych w procesie implementacji sama koncepcja \textit{Federated learning} wydaje się obiecująca. Wprowadzenie RODO (rozporządzenie o ochronie danych osobowych) wymusza ograniczenia związane z przetwarzaniem danych użytkowników. Fakt ten sprawia, że omawiana koncepcja zapewnia plusy dla obu stron. Klienci aplikacji otrzymują spersonalizowane rekomendacje bez przesyłania wrażliwych danych do systemu, natomiast właściciele aplikacji zwiększają skuteczność swojego modelu, odciążają własny sprzęt oraz nie muszą martwić się zabezpieczaniem dostępu do przechowywanych danych użytkowników. Jednak rozwiązanie to nie jest bez wad. Nie gwarantuje stuprocentowej skuteczności w zapewnieniu bezpieczeństwa danych użytkownika. Mimo, że dane zostają na urządzeniu użytkownika to nadal przesyłane są wytrenowane wagi modelu. Poprzez przesłane parametry istnieje ryzyko na odtworzenie danych wejściowych. W \cite{breakPrivacyFL} na przykładzie zdjęć wykorzystywanych jako dane wejściowe, pokazano, że istnieje możliwość odtworzenia obrazu. Oznacza to, że koncepcja wymaga jeszcze sporo czasu oraz pracy w celu zniwelowania swoich wad oraz odpowiedniego działania w środowiskach produkcyjnych.

\documentclass[a4paper,12pt,twoside,openany]{report}

\usepackage[utf8]{inputenc}
\usepackage[OT4]{fontenc}
\usepackage{polski}
\usepackage{multirow}
\usepackage{graphicx}
\graphicspath{ {./images/} }
\usepackage[nottoc]{tocbibind}
\usepackage{url}
\usepackage{subcaption}
\usepackage{float}
\usepackage{hyperref}
\usepackage{listings}

\newcommand{\specialcell}[2][c]{%
  \begin{tabular}[#1]{@{}c@{}}#2\end{tabular}}

\linespread{1.5} 


\begin{document}
\begin{titlepage}
    \begin{center}
        \large
        \textbf{Politechnika Poznańska}\\
        Wydział Informatyki i Telekomunikacji\\
        Instytut Informatyki\\
        \vspace{1.5cm}
            
        \textbf{Dominik Krystkowiak}\\
        Praca dyplomowa magisterska\\
        
        \vspace{1cm}
        \Huge
        \textbf{
        Analiza i zastosowanie zaawansowanych mechanizmów rekomendacji}
            
        \vfill
        \begin{flushright}\large
        Promotor dr inż. Anna Grocholewska-Czuryło\end{flushright}
        \vspace{0.8cm}
            
        \large
        Wrzesień 2020
            
    \end{center}
\end{titlepage}

\thispagestyle{empty}\vspace*{\fill}%
\begin{center}KARTA DYPLOMOWA\end{center}%
\vfill\cleardoublepage%

\thispagestyle{plain}
\begin{center}
    \textbf{Streszczenie}
\end{center}
Celem pracy jest przedstawienie oraz omówienie mechanizmów wykorzystywanych w systemach rekomendacyjnych. Wskazano potencjalne wyzwania, z którymi mierzą się systemy rekomendacyjne oraz pokazano ich wpływ na wydajność i skuteczność poszczególnych sposobów rekomendacji. Przedstawiono problem związany z bezpieczeństwem danych podczas procesu rekomendacji, a także omówiono zaproponowane rozwiązania z literatury. W pracy zaproponowano bezpieczny protokół uniemożliwiający na przechwycenie danych użytkowników z wykorzystaniem aktualnie badanych rozwiązań. Na koniec przeprowadzono eksperymenty związane z pojęciem \textit{Federated learning}.
\begin{center}
    \textbf{Abstract}
\end{center}
The aim of this master's thesis is to introduce and discuss the mechanisms that are used in recommender systems. Potential challenges that the recommendation systems are facing have been identified and their impact on the efficiency and effectiveness of the individual recommendation methods has been shown. The problem related to data security in the recommendation process has been introduced, also the proposed solutions from articles have been discussed. In the thesis secure protocol that prevents capturing the users' data using the actual solutions has been proposed. In the end experiments with Federated learning term have been done.

\tableofcontents* 
\cleardoublepage%


\chapter{Wstęp}

Rekomendacja (zgodnie z jedną z definicji zawartej w słowniku języka polskiego) oznacza “pozytywną opinię wydaną o kimś” \cite{pwn}. Każdy z nas w swoim życiu spotkał się z poleceniem przez znajomego ostatnio przeczytanej książki lub obejrzanego filmu. Na tej podstawie można wywnioskować, że pierwsze mechanizmy rekomendacji powstały na długo przed rozpowszechnieniem handlu internetowego, gdzie aktualnie zyskują coraz większą popularność.

W dobie ciągłego postępu oraz globalizacji systemy rekomendacyjne wchodzące w skład złożonych serwisów odgrywają coraz większą rolę. Dla niektórych serwisów systemy rekomendacyjne stały się nieodłączną częścią w utrzymaniu interakcji użytkownika z systemem na przykład: Youtube (rys. \ref{fig:yt}), Spotify (rys. \ref{fig:spotify}) czy HBO GO (rys. \ref{fig:hbo}). Wcześniej stosowane sposoby prezentacji (takie jak reklama czy newslettery) produktu stały się dla użytkowników zbyt natarczywe lub nie przynosiły firmom oczekiwanego rezultatu. Stąd korporacje i przedsiębiorstwa inwestują coraz większe pieniądze w implementację i wdrożenie systemów rekomendacyjnych.

\begin{figure}
    \centering
    \includegraphics[scale=0.85]{yt.JPG}
    \caption{Rekomendacje systemu Youtube.}
    \label{fig:yt}
\end{figure}

\begin{figure}
    \centering
    \includegraphics[scale=0.85]{spot.JPG}
    \caption{Rekomendacje systemu Spotify.}
    \label{fig:spotify}
\end{figure}

\begin{figure}
    \includegraphics[scale=0.4]{hbo.JPG}
    \caption{Rekomendacje systemu HBO GO.}
    \label{fig:hbo}
\end{figure}

\indent Głównym elementem pracy jest dokonanie przeglądu mechanizmów rekomendacji, podział ze względu na zastosowanie oraz opracowanie bezpiecznego protokołu oraz porównanie jego wydajności i skuteczności.

\indent Pierwszy rozdział przedstawia ogólną charakterystykę systemów rekomendacyjnych, ich cechy charakterystyczne oraz założenia. Ponadto przedstawiono najczęściej stosowane oraz sprawdzone komercyjnie rozwiązania podczas implementacji tego typu oprogramowania. Także omówiono problemy oraz wyzwania z którymi muszą zmagać się programiści tworząc nowoczesne systemy rekomendacyjne.  

\indent W kolejnym rozdziale został przedstawiony problem związany z bezpieczeństwem danych użytkownika oraz możliwością jego identyfikacji po wystawionych ocenach.

\indent W dalszej części pracy pokazano możliwe rozwiązanie problemu związanego z identyfikacją użytkownika.
W podsumowaniu omówiono otrzymane wyniki oraz możliwości udoskonalenia mechanizmu bezpieczeństwa 


\chapter{Przegląd systemów rekomendacyjnych}

Systemy rekomendacyjne pomagają użytkownikom w zdobyciu wartościowych i spersonalizowanych informacji z dużego źródła danych. Większość przypadków działania takiego systemu opiera się na przewidzeniu preferencji użytkownika na podstawie wcześniej zgromadzonych danych.

Ważnym momentem w historii systemów rekomendacyjnych był otwarty konkurs \textit{Netflix Prize} polegający na zaproponowaniu najlepszego algorytmu \textit{collaborative filtering} mającego na celu przewidzenie oceny filmu danej przez użytkownika, opierając się wyłącznie na poprzednich ocenach, bez dodatkowych informacjach o użytkownikach i filmach. W tym celu opublikowano zbiór testowy zawierający ponad 100 milionów ocen wystawionych przez 500 tysięcy anonimowych użytkowników. W wyzwaniu wzięło udział ponad 48 tysięcy zespołów z 182 państw [2].

Po konkursie \textit{Netflix Prize} głównym motywem przewodnim w pracach naukowych stała się faktoryzacja macierzy. Również w celu zwiększenia skuteczności algorytmu uwagę poświęcono takim zagadnieniom jak: wykorzystanie sieci społecznościowych oraz uczenia głębokiego w systemach rekomendacyjnych [1].

Rozdział omawia sposób działania systemów rekomendacyjnych, główne problemy i wyzwania stawiane tym systemom. Także przedstawiono popularne mechanizmy, sposoby ewaluacji oraz główne zalety poszczególnych rozwiązań.

\section{Założenia systemów rekomendacyjnych}

Głównymi zadaniami systemów rekomendacyjnych jest odfiltrowanie przedmiotów (na przykład: filmy, książki), które są dla użytkownika potencjalnie mało interesujące lub nieatrakcyjne. Użytkownik korzystający z serwisu wspieranego przez tego typu systemy otrzymuje jako proponowane, mały, spersonalizowany zestaw produktów z całego zbioru.

Systemy rekomendacyjne mogą zostać podzielone na dwie grupy ze względu na podejście [1]: spersonalizowane oraz niespersonalizowane. Pierwsze opierają swoje działanie na zebranych danych użytkownika. Informacje o odwiedzających mogą zostać zdobyte poprzez monitorowanie zachowań użytkowników lub poprzez bezpośrednie zapytanie użytkowników o ich preferencje. Natomiast podejście niespersonalizowane bazuje na zachowaniach pozostałych użytkowników.

Dane zazwyczaj przedstawiane są w formie dwuwymiarowej macierzy $R = U \times I$, gdzie U jest to zbiór n użytkowników oraz I zbiór m przedmiotów. Tabela \ref{macierzUI} przedstawia przykład takiej macierzy, wiersze reprezentują użytkowników, kolumny przedmioty, oceny są w przedziale 1 do 5, natomiast 0 oznacza brak oceny. Aby uzyskać lepsze rekomendacje systemy mogą wykorzystać macierze o większej liczbie wymiarów (zawierających czas, lokalizację czy informacje uzyskane z mediów społecznościowych).

Dzięki coraz lepszym technikom zdobywania danych oraz spadającej tendencji do udzielania ocen wśród klientów, popularnym stało się wykorzystanie domniemanych danych (\textit{ang. implict data}) takich jak: liczba kliknięć lub wyświetleń do budowania akademickich oraz branżowych systemów rekomendacyjnych. Dane takie w przeciwieństwie do informacji sprecyzowanych (\textit{ang. explict data}) nie opierają się o negatywne sprzężenie zwrotne oraz nie wymagają metryki pewności (nawiązują do częstotliwości występowania zjawiska).

\label{macierzUI}
\begin{table}[h]
\centering
\caption{Przykład macierzy użytkownik-przedmiot.}
\begin{tabular}{|c|c|c|c|c|c|}
\hline
&
Przedmiot1 &
Przedmiot2 &
Przedmiot3 &
Przedmiot4 &
Przedmiot5
\\
\hline
  
Użytkownik1 &
1 &
5 &
4 &
2 &
3
\\
\hline
  
Użytkownik2 &
5 &
0 &
2 &
4 &
0
\\
\hline

Użytkownik3 &
1 &
0 &
4 &
0 &
4
\\
\hline

Użytkownik4 &
3 &
0 &
3 &
1 &
5
\\
  \hline

Użytkownik5 &
2 &
0 &
2 &
0 &
3
\\
\hline
\end{tabular} 
\end{table}

Ze względu na rodzaj wyjścia (otrzymanego wyniku) systemy rekomendacyjne możemy podzielić na następujące dwie grupy: pojedyncza predykcja oraz top-N predykcji. Pierwsza polega na przewidzeniu dokładnej oceny, którą użytkownik mógłby dać przedmiotowi, natomiast drugie podejście zwraca N nieocenionych przedmiotów, które mogłyby najbardziej przypaść do gustu użytkownikowi.


\section{Problemy i wyzwania}

Rzadkość danych - następuje w momencie, gdy wielu użytkowników oceniło tylko kilka przedmiotów przez co system rekomendacyjny nie może poznać preferencji użytkownika (macierz użytkownik-przedmiot ma zbyt wiele pustych wartości).

Skalowalność - systemy rekomendacyjne wraz ze wzrostem liczby użytkowników muszą radzić sobie z napływem coraz większej liczby danych do przetworzenia. Gdy takie dane wynoszą miliony użycie standardowych mechanizmów rekomendacyjnych może powodować otrzymanie wyników rekomendacji z dużym opóźnieniem (niewystarczający krótki czas na uzyskanie odpowiedzi). Bardziej złożony system wymaga większej liczby osób usprawniających oraz zarządzających jego działaniem, co równa się większemu kosztowi utrzymania zarówno pracowników jak i sprzętu komputerowego na którym taki system jest wdrożony.
    
Czarna owca - zjawisko występuje w momencie, gdy preferencje jednego z użytkowników nie pokrywają się z wyodrębnionymi grupami użytkowników, stąd użytkowników nie będzie w stanie otrzymać spersonalizowanych rekomendacji.

Długi ogon - następuje w momencie, gdy rekomendacja opiera się na produktach podobnych do siebie. Wtedy użytkownicy oglądają tylko część oferowanych przez serwis przedmiotów. Zjawisko to objawia się tym, że produkty, które potencjalnie mogą się podobać konsumentowi nie zostaną mu zaprezentowane.

Zimny start (ang \textit{cold-start)} - występuje w momencie rejestracji do systemu nowego użytkownika lub przedmiotu. Przedmioty nie mogą zostać zarekomendowane, ponieważ nie mają żadnej oceny natomiast nowi klienci dokonali zbyt mało ocen.

\section{Content-based filtering}
\textit{Content-based filtering} oparte jest na założeniu, że użytkownik polubi przedmioty o podobnej charakterystyce do tych, które poprzednio zostały przez niego polubione (schemat działania przedstawiony na rys. \ref{fig:content-based}). Do tego celu wykorzystywane są opisy produktów (na przykład przedstawione w formie tagów) do generowania rekomendacji. Plusem tego podejścia jest fakt, że do przeprowadzenia rekomendacji nie są potrzebne dane użytkownika, jednak wymagany jest dokładny opis oraz cechy charakterystyczne danego produktu. Przykładami takich cech może być zawartość książki lub sygnał akustyczny utworu muzycznego.

\begin{figure}
    \centering
    \includegraphics{images/content-based.png}
    \caption{Sposób działania mechanizmu \textit{Content-based filtering}.}
    \label{fig:content-based}
\end{figure}

\section{Collaborative filtering}

Podejście to nie wykorzystuje cech charakterystycznych przedmiotu (tak jak to było w metodzie \textit{Content-based filtering}), jednak problemem jest wygenerowanie rekomendacji dla nowego użytkownika lub produktu (\textit{cold start}). Metoda ta stosowana jest w dużych, komercyjnych aplikacjach. Ma wiele dobrze poznanych, różnorodnych algorytmów i wariacji przez co stosowana jest w wielu domenach.

Opisywana metoda może zostać podzielona na dwie podkategorie:

\begin{itemize}
    \item podejście pamięciowe (\textit{ang. memory based}) rekomenduje przedmiot na podstawie zbioru poprzednio ocenionych przez użytkownika przedmiotów. Dla przykładowego użytkownika \textit{u}, predykcja wyliczana jest na podstawie podobnych do niego użytkowników (główna idea przedstawiona na rys. \ref{fig:collaborative}) lub przedmiotów podobnych do tych, które wcześniej ocenił. Miara podobieństwa najczęściej obliczana jest korelacją Pearsona lub podobieństwem kosinusowym,
    \item podejście modelowe (\textit{ang. content based}) wykorzystuje dane historyczne. Jedną z najczęściej stosowanych metod w tym podejściu jest faktoryzacja macierzy, która zakłada, że użytkownicy oceniają przedmioty na podstawie pewnych cech. W przypadku muzyki może być jej gatunek lub wykonawca. W takim przypadku system rekomendacyjny jest w stanie odkryć czynniki umożliwiające na lepszą skuteczność systemu (liczba czynników powinna być mniejsza od liczby użytkowników i przedmiotów). 
\end{itemize}

W celu określenia miary podobieństwa pomiędzy użytkownikami najczęściej stosowana jest korelacja Pearsona. Definiowana jest ona w następujący sposób:

\begin{equation}
Pearson(x,y) = \frac{\sum\limits_{i=1}^n(x_i - \overline{x})(y_i - \overline{y})}{\sqrt{\sum\limits_{i=1}^n(x_i - \overline{x})^2}\sqrt{\sum\limits_{i=1}^n(y_i - \overline{y})^2}},
\end{equation}
dla podobieństwa dwóch użytkowników wzór wygląda następująco:

\begin{equation}
Pearson(a,b) = \frac{\sum\limits_{p \in P}(r_{a,p} - \overline{r}_a)(r_{b,p} - \overline{r}_b)}{\sqrt{\sum\limits_{p \in P}(r_{a,p} - \overline{r}_a)^2}\sqrt{\sum\limits_{p \in P}(r_{b,p} - \overline{r}_b)^2}},
\end{equation} gdzie:
\begin{itemize}
    \item a,b -- użytkownicy,
    \item $r_{a,p}$ -- ocena użytkownika a dla przedmiotu p,
    \item $r_{b,p}$ -- ocena użytkownika b dla przedmiotu p,
    \item $\overline{r}_a$ -- średnia ocen użytkownika a,
    \item $\overline{r}_b$ -- średnia ocen użytkownika b,
    \item P -- zbiór przedmiotów ocenionych przez obu użytkowników.
\end{itemize}

Uzyskane w ten sposób wartości są z przedziału [-1; 1], im wyższa liczba tym użytkownicy są do siebie bardziej podobni.

\begin{figure}
    \centering
    \includegraphics[scale=0.7]{images/collaborative.png}
    \caption{Sposób działania mechanizmu \textit{Collaborative filtering}.}
    \label{fig:collaborative}
\end{figure}

\section{Knowledge-based filtering}

Podejście oparte o wiedzę swoje działanie na wykorzystaniu źródeł wiedzy, które nie zostały uzyskane za pomocą poprzednio omawianych metod. Rekomendacje generowane są na podstawie wiedzy dziedzinowej preferencji użytkownika, cech przedmiotów oraz jak te cechy mogą spełniać potrzeby użytkownika. Wyróżnia się dwa podejścia:
\begin{itemize}
    \item oparte o przypadek (ang. \textit{case-based}) - używają  metryk podobieństwa do uzyskania przedmiotów zgodnych z potrzebami użytkownika,
    \item oparte o ograniczenia (ang. \textit{constraint-based}) - opiera działanie na zbiorze reguł, aby znaleźć przedmioty spełniające wymagania użytkowników.

\end{itemize}

Oba podejścia wymagają od użytkownika podania wymagań na podstawie, których system będzie próbował otrzymać odpowiedź na żądanie użytkownika.

\section{Podejście hybyrdowe}

Podejście hybrydowe ma na celu zniwelować wady pojedynczych metod poprzez połączenie dwóch lub więcej mechanizmów. Najczęściej stosowaną kombinacją jest \textit{Content-based} wraz z \textit{Collaborative filtering}. 

Koncepcja \textit{Content-Boosted Collaborative Filtering} wykorzystuje pseudowektory ocen użytkowników $V_{u}$, który zawiera oceny użytkownika $u$ oraz predykcję ocen przedmiotów nieocenionych przez niego. 

\begin{equation}
V_{u,i} = \left\{ \begin{array}{ll}
\textrm{$r_{u,i}$ :} & \textrm{gdy użytkownik u ocenił przedmiot i}\\
\textrm{$c_{u,i}$ :} & \textrm{w przeciwnym przypadku}
\end{array} \right.
\end{equation} gdzie:
\begin{itemize}
    \item $r_{u,i}$ - ocena podana przez użytkownika u,
    \item $r_{u,i}$ - przewidywana ocena przez czysty system typu \textit{content-based}. 
\end{itemize}
Tak utworzone pseudowektory dla wszystkich użytkowników tworzą macierz V na której zostanie przeprowadzone filtrowanie kolaboracyjne.

\section{Sposoby ewaluacji}\label{metryki}

Ważnym aspektem podczas tworzenia systemów rekomendacyjnych jest ocena działania zaimplementowanego algorytmu. Ze względu na środowisko można je podzielić na testy offline oraz online lub na statystyczne (np. MAE) i wspierające decyzje (np. AUC) [Herlocker et al.1999] Ewaluacja offline jest najprostszym podejściem, ponieważ nie wymaga interakcji z rzeczywistym odbiorcą systemu. Natomiast drugie podejście daje lepsze rezultaty jednak wymaga to dużych nakładów finansowych oraz w niektórych przypadkach ciężko zrozumieć powiązania pomiędzy użytkownikami a właściwościami systemu. Również wyróżnia się testy na małej liczbie użytkowników w kontrolowanym środowisku oraz raportowanie ich doświadczeń podczas pracy z systemem.

Jakość predykcji w systemach rekomendacyjnych mierzona jest poprzez następujące miary:

Średni błąd kwadratowy (\textit{ang. Mean Absolute Error}) - mierzy bezwzględny błąd pomiędzy wyestymowaną wartością, a prawdziwą.

\begin{equation}
    MAE = \frac{\sum\limits_{i,j\in{k}}|\hat{r}_{u,i} - r_{u,i}|}{N},
\end{equation} gdzie:
\begin{itemize}
    \item $\hat{r}_{u,i}$ - predykcja oceny użytkownika u dla przedmiotu j,
    \item $r_{u,i}$ - rzeczywista ocena,
    \item N - liczba wszystkich ocen w danych testowych.
\end{itemize}

Pierwiastek średniego błędu kwadratowego (\textit{ang. Root Mean Square Error}) - wzmacnia błąd średniokwadratowy pomiędzy przewidywaną wartością, a rzeczywistą. Sprawdza się, gdy błąd ma duży wpływ na decyzje użytkowników.

\begin{equation}
    MAE = \sqrt{\frac{\sum\limits_{i,j\in{k}}(\hat{r}_{u,i} - r_{u,i})^2}{N}},
\end{equation}

Precyzja (ang. \textit{Precision}) jest to stosunek prawidłowo sklasyfikowanych elementów do wszystkich otrzymanych.

\begin{equation}
    Precision = \frac{T_p}{T_p + F_p},
\end{equation} \\
Czułość (ang. \textit{Recall}) jest to stosunek prawidłowo sklasyfikowanych elementów do wszystkich, które powinny zostać rozpoznane.
\begin{equation}
    Recall = \frac{T_p}{T_p + F_n},
\end{equation} \\

\begin{table}[h]
\centering
\caption{Macierz błędów.}
\begin{tabular}{cc|p{2cm}|p{2cm}|}
  \cline{3-4}
  & & \multicolumn{2}{ |c| }{Klasa rzeczywista}
  \\
  \cline{3-4}
  & & pozytywna & negatywna \\ \cline{1-4}
  \multicolumn{1}{ |c  }{\multirow{2}{*}{Klasa predykowana} } & \multicolumn{1}{ |c| }{pozytywna} & prawdziwie pozytywna ($T_p$) & fałszywie pozytywna ($F_p$)     \\ \cline{2-4}
  \multicolumn{1}{ |c  }{}                        &
  \multicolumn{1}{ |c| }{fałszywa} & fałszywie negatywna ($F_n$) & prawdziwie negatywna ($T_n$)     \\ \cline{1-4}
    \label{macierzBledow}
\end{tabular} 
\end{table}

F1 - średnia harmoniczna dla wartości precyzji i czułośći. 

\begin{equation}
    F1 = \frac{2 * precision * recall}{precision + recall},
\end{equation}

Pole pod krzywą ROC (\textit{ang. Receiver operating Characteristic}) - krzywa ROC pokazuje wydajność klasyfikatora binarnego na dwuwymiarowej płaszczyźnie. Pole pod krzywą ROC jest wykorzystywane do mierzenia zdolności systemu do odróżniania dobrych predykcji od złych. Większa wartość tej metryki oznacza lepszą wydajność systemu. Macierz błędów została przedstawiona w tabeli \ref{macierzBledow}.

[DO ZROBIENIA offiline
train/cross-validation/test
różne wagi (klasyfikacyjne vs rankingowe)

online
algorytm dziala na produkcji
CTR GMV - podstawowe marketingowe miary
miary offline warto obliczyc dla danych splywajacych z produkcji
testy a/b]

\section{Skuteczność metod}

WSTEP

Systemy rekomendacyjne wykorzystujące podejście \textit{collaborative-filtering} uzyskują zadowalające wyniku związane z wydajnością, jednak nie są odporne na problem zimnego startu. Także problemem może okazać się rzadkość danych, jeżeli wartość wynosi około 0.5 lub więcej wtedy należy rozważyć użycie innego rozwiązania.

\begin{table}[h]
\centering
\caption{Przykład macierzy użytkownik-przedmiot.}
\begin{tabular}{|c|c|c|c|c|c|}
\hline
&
\specialcell{Rzadkość\\danych} &
Skalowalność &
\specialcell{Czarna\\owca} &
\specialcell{Długi\\ogon} &
\specialcell{Zimny\\start}
\\
\hline
  
Content-based &
 &
 &
 &
 &

\\
\hline
  
Collaborative &
- &
 &
 &
 &
-
\\
\hline

Knowledge-based &
 &
 &
 &
 &

\\
\hline

\specialcell{Podejście\\hybyrdowe} &
 &
 &
 &
 &

\\
\hline
\end{tabular} 
\label{tabelaMetProblem}
\end{table}

\chapter{Aspekt bezpieczeństwa w systemach rekomendacyjnych}
Systemy rekomendacyjne muszą gromadzić wiele danych o użytkownikach w celu wykonania trafnego przewidywania potencjalnych obiektów zainteresowania. W zależności od metody serwis dokonuje porównania  profili użytkowników, oblicza podobieństwo pomiędzy użytkownikami, czy grupuje użytkowników o zbliżonych cechach. Gromadzone w tym celu dane mogą pozwolić na identyfikację użytkownika, przez co twórcy systemu muszą wybierać pomiędzy prywatnością użytkowników, a skutecznością w dokonywaniu rekomendacji. Taki system, z wrażliwymi danymi użytkowników, może stać się celem grup hakerskich chcących pozyskać tak cenne informacje. Stąd zapewnienie bezpieczeństwa podczas przesyłania danych pomiędzy serwisami oraz dokonywania obliczeń na nich powinno stać się priorytetem. Jednak większość prac skupia się na zwiększeniu skuteczności czy wydajności \cite{recent_developments}.

\section{Prywatność}
Prywatność w kontekście systemów rekomendacyjnych oznacza, że podczas pracy systemy żadne dane nie mogą wyciec, a także otrzymana predykcja nie powinna pozwolić na identyfikację użytkownika. Na rysunku \ref{fig:podatnosci} przedstawiono potencjalne miejsca, w których dane mogą zostać przechwycone. Jak widać, atakujący mogą uzyskać dane podczas wprowadzania ich do systemu, przesyłania ich pomiędzy poszczególnymi komponentami usługi, jak i w trakcie wysłania predykcji do użytkownika. Idealny, zachowujący prywatność system rekomendacyjny powinien być bezpieczny bez znaczącej utraty wydajności jako całości, a przede wszystkim, bez zmniejszenia trafności rekomendacji. Systemy rekomendacyjne muszą stawić czoła ochronie prywatności użytkowników jak i ochronie wszystkich uczestników procesu rekomendacji. Wiele technik uczenia maszynowego w celu zachowania prywatności skupia się na wykorzystaniu wielu części do wspólnego trenowania modelu bez dzielenia się swoimi danymi w oryginalnej formie (dane są przesyłane wyłącznie w sposób zaszyfrowany). Osiąga się to poprzez:
\begin{itemize}
    \item perturbacje danych (ang. \textit{data perturbation}),
    \item bezpieczne obliczenia wieloczęściowe (ang. \textit{secure multiparty computation}).
\end{itemize}

\begin{figure}
    \centering
    \includegraphics[scale=0.85]{images/podatnosci.png}
    \caption{Potencjalne podatności systemu.}    Źródło: opracowanie własne na podstawie \cite{practicalPrivacy}

    \label{fig:podatnosci}
\end{figure}
\section{Perturbacja danych}
Właściciel danych (użytkownik) dokonuje perturbacji własnych danych poprzez dodanie losowego szumu, a następnie przesyła je w celu dalszego przetwarzania.
Możemy wyróżnić następujące rodzaje perturbacji danych:
\begin{itemize}
    \item addytywne - opisane następującym wzorem 
    Y = X + C, gdzie C jest szumem dodanym do oryginalnej macierzy X. W praktyce każdy wiersz C generowany jest niezależnie,
    \item multiplikatywne - opisane wzorem Y = MX, gdzie M jest to macierz przez którą przekształcana są dane źródłowe. W przypadku tej metody nie ma gwarancji, że zostanie zachowana prywatność (przypadek w którym atakujący zna porcję danych w zbiorze X oraz ich odpowiednik w zbiorze M),
    \item prywatność różnicowa - wyróżnia się trzy podejścia w zależności od momentu wprowadzenia szumu: do danych wejściowych, w trakcie przetwarzania lub do danych wyjściowych.
\end{itemize}

Natomiast ze względu na miejsce gdzie dochodzi i jak do perturbacji danych podział wygląda następująco \cite{PPML}:
\begin{itemize}
    \item perturbacja wejścia - szum dodawany jest do danych,
    \item perturbacja algorytmu - dodanie szumu dochodzi pomiędzy iteracjami danego algorytmu,
    \item perturbacja wyjścia - na danych wejściowych uruchamiany jest algorytm niezachowujący prywatności, a następnie do uzyskanego rezultatu dodawany jest szum,
    \item obiektywna perturbacja - szum dodawany jest do funkcji obiektywnej (funkcja używana do optymalizacji problemu).
\end{itemize}

Głównym problemem w wykorzystaniu perturbacji danych w celu zapewnienia prywatności jest brak gwarancji, że uzyskane wyniki będą tak samo trafne, jak w przypadku bez wprowadzenia szumu w danych. Dodanie zbyt dużego szumu może powodować niską jakość wskazywania podobnych użytkowników lub przedmiotów.



\section{Bezpieczne obliczanie wieloczęściowe}

Metoda pozwala rozproszonym stronom na wspólne obliczanie, np. klucza, sekretu itp., bez konieczności odkrywania ich prywatnych danych wejściowych oraz wyjściowych. Ponadto zastosowanie obliczania wieloczęściowego nie wymaga użycia zaufanej trzeciej strony (bezpieczeństwo pozostaje takie samo jak w przypadku wykorzystania trzeciej strony). 
Wyróżniane są następujące rozwiązania:
\begin{itemize}
    \item rozwiązania oparte na bezpiecznym dodawaniu wektorów (\textit{ang. Secure Vector Addition based Solutions}),
    \item szyfrowanie homomorficzne - pozwala na wykonywanie operacji (dodawanie, mnożenie) na zaszyfrowanych danych, bez konieczności ich deszyfrowania. Ze względu na wysoki koszt związany z wykonywanymi operacjami, w modelach uczenia maszynowego wykorzystuje się częściej dodawanie,
    \item podejścia oparte na bezpiecznych produktach skalarnych (\textit{ang. Secure scalar product based approach}) - umożliwia na obliczanie przez strony iloczynu skalarnego ich prywatnych wektorów, bez ich przekazywania. Metoda wykorzystywana przede wszystkim w celu eksploracji danych oraz analizie statystycznej,
    \item podejścia oparte o "zniekształcony obwód" (\textit{ang. Garbled circuit based approach}) - jedna ze stron przesyła funkcję obliczeniową w postaci obwodu, w ten sposób druga strona może użyć tej samej funkcji w celu wykonania obliczeń na własnych danych.
\end{itemize}

W celu wykorzystania bezpiecznego obliczania wieloczęściowego powstały specjalne bibliotek takie jak: CrypTFlow (nakładka na bibliotekę Tensorflow składająca się z trzech komponentów - Athos, Porthos i Aramis) \cite{kumar2020cryptflow} czy EzPC \cite{chandran2019ezpc}.

Z powodu wysokiego kosztu obliczeniowego bezpieczne obliczanie wieloczęściowe nie zawsze znajdują zastosowanie w komercyjnych rozwiązaniach związanych z rekomendacją. Spowodowane jest tym, że metody te na ten moment nie mogą być rozpatrywane przy systemach wymagających działania w czasie rzeczywistym. \cite{secureMultipartyComputation}

\section{Federated learning}
\label{section:federatedLearning}

Pojęcie wprowadzone przez firmę Google w 2016 roku. Standardowe uczenie maszynowe zakłada, że dane treningowe znajdują się na jednej maszynie lub w centrum danych. Model \textit{Federated learning} umożliwia na wspólne uczenie z udostępnianego wcześniej wyuczonego modelu bez dzielenia się prywatnymi danymi z innymi użytkownikami bądź z scentralizowanym serwerem, a także nie wymaga przechowywania danych po stronie serwera.

Uproszczony opis działania algorytmu (na rys. \ref{fig:fl} przedstawiono schemat):
\begin{enumerate}
    \item Urządzenie (na przykład smartphone) pobiera wcześniej wytrenowany model.
    \item Model jest aktualizowany (ponownie trenowany na danych znajdujących się na urządzeniu) (A).
    \item Aktualizacja modelu wysyłana jest do chmury poprzez zaszyfrowaną komunikację.
    \item Aktualizacja jest uśredniana z pozostałymi aktualizacjami modelu otrzymanymi od innych użytkowników (B).
    \item Model jest aktualizowany na bazie uśrednionych wyników (C).
\end{enumerate}

\begin{figure}
    \centering
    \includegraphics[scale=0.6]{images/fl.png}
    \caption{Schemat działania \textit{Federated Learning}.}    Źródło: Google AI Blog \cite{fedderatedLearning}
    \label{fig:fl}
\end{figure}

\textit{Federated learning} wykorzystywane jest tam gdzie dane są wrażliwe lub ilość danych jest na tyle duża, że przesłanie ich do jednego zbiorczego kontenera jest nieopłacalne. Przykładowymi zastosowaniami są między innymi w \textit{Google GBoard} \cite{fedderatedLearning} - wirtualna klawiatura przewidująca kolejne słowa użytkownika, w badaniach farmaceutycznych \cite{fedderatedLearningHealth} czy w samochodach autonomicznych \cite{fedderatedLearningVehicle}.

Zastosowanie tego rozwiązania umożliwia na zmniejszenie czasu przetwarzania danych poprzez nie wysyłanie ich w stanie surowym do serwera, a procesowanie danych bezpośrednio na urządzeniu, które zebrało dane. Zastosowanie tej technologii nie wymaga stałego połączenia z Internetem, a także nie wymaga dodatkowej infrastruktury sprzętowej do sprawnej komunikacji czy wykonywania obliczeń

Mimo szeregu zalet zastosowanie tego modelu wymaga częstej komunikacji pomiędzy poszczególnymi węzłami uczącymi. Oprócz dobrej wydajności sprzętu i pamięci, ważne jest także zapewnienie wysokiej przepustowości łącza w celu wymiany uzyskanych wyników. Kolejną wadą jest fakt, że nie ma pewności co do jakości danych na których generowany jest model. Jeżeli te dane skrajnie odbiegają od pozostałych może to spowodować znaczącą zmianę głównego modelu.

\section{Przegląd aktualnych rozwiązań}

W \cite{practicalPrivacy} zaprezentowano schemat składający się z dwóch faz. W pierwszej fazie użytkownicy szyfrują swoje oceny oraz przesyłają szyfrogram do serwera. Serwer korzystając z właściwości homomorficznych (obliczenia na zaszyfrowanych danych) oblicza podobieństwa i średnie, które następnie są przechowywane w bazie. Druga faza to generowanie rekomendacji w której użytkownik na podstawie wysłanych zaszyfrowanych danych otrzymuje od serwera rekomendacje, które może odszyfrować wyłącznie on za pomocą prywatnego klucza. Proponowane rozwiązanie dotyczy zarówno metody \textit{Content-based Filtering} jak i \textit{Collaborative filtering}.

Inne mechanizmy zaproponowano w \cite{contributionsToSecurityRS}. W pierwszym podejściu ograniczono ilość wymaganych danych przesyłanych przez użytkownika, co powoduje niestety zmniejszenie liczby wymiarów, które są wykorzystywane do rekomendacji. Drugim podejściem jest protokół komunikacyjny umożliwiający na obliczenie podobieństwa użytkowników bazując na dowodach o wiedzy zerowej (jedna ze stron może udowodnić, że posiada daną informację bez jej ujawniania). Podczas wymiany dwóch użytkowników starają się wyłącznie dowiedzieć czy są do siebie podobnie. Według autora zaprezentowane podejścia mogą być ze sobą połączone jednocześnie.

Natomiast w \cite{weakTies} przedstawiono zastosowanie \textit{weak ties}, które są nieoczywistymi połączeniami między użytkownikami dające trafne rekomendacje. W tym celu zastosowano model \textit{jumping connections}, który dokonuje rekomendacji poprzez serię skoków pomiędzy użytkownikami (w najbliższym sąsiedztwie rekomendacji występuje jedynie jeden skok).



\chapter{Model systemu rekomendacyjnego chroniącego prywatność}

W rozdziale zaproponowanie rozwiązania zwiększającego bezpieczeństwo oraz zachowanie prywatnych danych użytkownika. W tym celu wykorzystano \textit{Federated learning} omówiony w rozdziale \ref{section:federatedLearning} wraz z dodatkowymi zabezpieczeniami mającymi na celu zniwelować przedstawione problemy związane z tym rozwiązaniem. Także przedstawiono jak mógłby wyglądać proces przygotowania takiej metody dla nowo-powstałego systemu jak i dla już zaimplementowanego systemu z wykorzystaniem dostępnych bibliotek i technologii.

\section{Projekt rozwiązania}

W proponowanym rozwiązaniu wyróżniono trzech aktorów. Pierwszy z nich, zwykły użytkownik uczestniczy w normalnym przebiegu zaproponowanym w rozwiązaniu \textit{Federated learning}. Użytkownik na podstawie swoich danych poprawia otrzymany przez serwer model, następnie bez ujawniania swoich danych przekazuje dalej wyłącznie poprawiony przez siebie model.

Jednym z potencjalnych problemów może być założenie, że użytkownicy będą uczciwie uczestniczyć w całym omawianym procesie. Oznacza, to że użytkownik może zmniejszyć skuteczność modelu rekomendacji poprzez umieszczenie w nich niepoprawnych danych (niezgodnych z rzeczywistością). W przypadku pojedynczego użytkownika, który próbuje zakłócić pracę systemu istnieje mechanizm uśredniający otrzymane wyniki od wszystkich użytkowników. Jednak w przypadku zorganizowanego ataku z wielu kont wyłącznie normalizacja może nie być wystarczająca. W celu poradzenia sobie z tym problemem wprowadzono drugi typ aktora - użytkownik niezaufany, który ma możliwość wyłącznie otrzymania modelu od serwera głównego. Jest on wyłączony z usprawnienia udostępnianego modelu (nie wysyła on swoich rezultatów uczenia). Tego typu użytkownik po określonym czasie użytkowania systemu, ilości wystawionych ocen lub po odpowiednim procesie weryfikacyjnym może zostać przemianowany na zwykłego użytkownika uczestniczącego w procesie usprawniania wspólnego modelu.

Kolejnym problemem może być koszt nauki na systemach mobilnych. Niektórzy użytkownicy mogą posiadać słabszy sprzęt (zasoby sprzętowe czy bateria), który może uniemożliwić ponowne uczenie otrzymanego modelu na urządzeniu. W przypadku wolnego łącza lub jego braku, problemem może być pobieranie lub odesłanie modelu do głównego serwera. W tym celu użytkownik z ograniczonymi zasobami zostałby zmuszony do przesłania swoich zaszyfrowanych danych do serwera, tam wykorzystując właściwości szyfrowania homomorficznego model zostałby zaktualizowany, a rekomendacje przesłane do użytkownika.

\section{Fazy budowania systemu rekomendacyjnego}

Podczas implementacji systemu rekomendacyjnego wykorzystano typowy przepływ (przedstawiony na rysunku \ref{fig:rs_pipeline}) używany do tworzenia tego typu serwisów, który składa się z następujące pięć faz \cite{rs_in_real}:
\begin{itemize}
    \item wstępne przetwarzanie - na ten etap składają się czynności związane z transformacją danych do macierzy użytkownik-produkt oraz normalizacja w celu spłaszczenia wartości odstających (użytkownicy, którzy są nad wyraz pozytywni oraz negatywni w stosunku do dawania ocen),
    \item trenowanie - proces budowania modelu,
    \item optymalizacja hiper parametrów - wielokrotne trenowanie w celu dostrojenia parametrów w taki sposób, aby uzyskać jak najlepsze wyniki,
    \item przetwarzanie końcowe - sortowanie danych w celu uzyskania N najlepszych rekomendacji dla użytkownika, filtrowanie oraz wykluczenie wcześniej zakupionych lub negatywnie ocenionych przedmiotów,
    \item ewaluacja - testowanie stworzonego modelu poprzez ukrywanie/maskowanie ocen, a następnie użycie miar do ewaluacji (szczegółowo opisane w podrozdziale \ref{metryki}).
\end{itemize}{}

\begin{figure}
    \includegraphics[scale=0.85]{rs_pipeline.png}
    \caption{Fazy przepływu podczas tworzenia systemu rekomendacyjnego.}
    \label{fig:rs_pipeline}
\end{figure}

\section{Dane testowe}

\section{Przegląd dostępnych technologii}

Istnieje wiele dostępnych bibliotek czy frameworków umożliwiające na implementacje rozwiązań typu \textit{Federated learning}. Większość z nich umożliwia na napisanie kodu w języku Python. Jednymi z bardziej popularnych oraz dobrze udokumentowanych są następujące technologie:

\begin{itemize}
    \item PySyft - narzędzie opracowane przez grupę OpenMinded. PySyft rozszerza popularne biblioteki takie jak: PyTorch, Tensorflow oraz Keras o możliwość zdalnego wykonywania, szyfrowania homomorficznego czy obliczenia z wykorzystaniem wielu użytkowników,
    \item Tensorflow Federated - biblioteka wykorzystywana do uczenia maszynowego i innych obliczeń na danych nie scentralizowanych oparta o otwarty kod źródłowy. Stworzona w celu badań oraz eksperymentowania z koncepcją \textit{Federated learning}.
\end{itemize}

\section{Fragmenty implementacji}


\chapter{Podsumowanie}

Cele postawione przed pisaniem pracy udało się zrealizować. Przedstawiono najczęściej stosowane algorytmy stosowane podczas tworzenia systemów rekomendacyjnych, główne założenia, problemy i wyzwania, oraz sposoby ewaluacji zaprojektowanych rozwiązań. Pierwsza część pracy wskazuje, że wykorzystanie poszczególnych metod zależy nie tylko od danych jakie gromadzone są przez system, ale również od liczby użytkowników czy specyfikacji (założeń) danego systemu. Połączenie poszczególnych rozwiązań w jedno umożliwia na pozbycie się występujących problemów.

Dalsza część pracy przedstawia problem bezpieczeństwa danych w szeroko pojętych systemach rekomendacyjnych oraz sposobów zwiększających ochronę danych użytkowników podczas procesu przetwarzania danych. Zaproponowany protokół bezpiecznej rekomendacji wykorzystuje opisane w pracy metody (szyfrowanie homomorficzne oraz \textit{Federated learning}) mając na uwadze bezpieczeństwo zarówno klientów jak i samej aplikacji. Rozwiązanie miało na celu zniwelowanie wad metod działających oddzielnie poprzez ich połączenie - takich jak ograniczone zasoby sprzętowe i programowe czy wydłużenie procesu rekomendacji.

Dalsze eksperymenty miały na celu udowodnienie, że wykorzystanie stosunkowo nowego podejścia w uczeniu nie wpływa na pogorszenie skuteczności rekomendacji lub czasu wymaganego do przetworzenia otrzymanych danych. W przypadku wymaganego czasu nie ma to dużego wpływu na działanie systemu, ponieważ sam proces uczenia odbywa się na urządzeniu użytkownika bez ujawniania żadnych danych. Same opóźnienia nie miałyby dużego oddziaływania na działanie całego systemu. Jednak duży wpływ na funkcjonowanie aplikacji ma skuteczność uzyskana przez klientów. Przeprowadzone eksperymenty pokazują, że zastosowanie metody \textit{Federated learning} nie powoduje znacznego spadku otrzymanych rekomendacji. 

Jednak w trakcie implementacji okazało się, że testowane biblioteki nie są jeszcze w takim stanie, aby były wykorzystywane w produkcyjnie działających aplikacji. Spowodowane jest to małą liczbą przykładów zawartych w dokumentacji, stosunkowo małą społecznością użytkowników dzielących się wiedzą oraz wczesną wersją samych bibliotek (w momencie pisania pracy jest to odpowiednio wersja 0.16.1 dla \textit{Tensorflow federated} oraz wersja 0.2.8 dla biblioteki \textit{PySyft}). Wczesne wersje mogą być niestabilne, a także w trakcie rozwoju może dojść do znaczących zmian powodujących brak kompatybilności wstecznej.

Mimo licznych problemów napotkanych w procesie implementacji sama koncepcja \textit{Federated learning} wydaje się obiecująca. Wprowadzenie RODO (rozporządzenie o ochronie danych osobowych) wymusza ograniczenia związane z przetwarzaniem danych użytkowników. Fakt ten sprawia, że omawiana koncepcja zapewnia plusy dla obu stron. Klienci aplikacji otrzymują spersonalizowane rekomendacje bez przesyłania wrażliwych danych do systemu, natomiast właściciele aplikacji zwiększają skuteczność swojego modelu, odciążają własny sprzęt oraz nie muszą martwić się zabezpieczaniem dostępu do przechowywanych danych użytkowników. Jednak rozwiązanie to nie jest bez wad. Nie gwarantuje stuprocentowej skuteczności w zapewnieniu bezpieczeństwa danych użytkownika. Mimo, że dane zostają na urządzeniu użytkownika to nadal przesyłane są wytrenowane wagi modelu. Poprzez przesłane parametry istnieje ryzyko na odtworzenie danych wejściowych. W \cite{breakPrivacyFL} na przykładzie zdjęć wykorzystywanych jako dane wejściowe, pokazano, że istnieje możliwość odtworzenia obrazu. Oznacza to, że koncepcja wymaga jeszcze sporo czasu oraz pracy w celu zniwelowania swoich wad oraz odpowiedniego działania w środowiskach produkcyjnych.


\medskip

\bibliographystyle{plamsplain}
\bibliography{bibliography}

\end{document}

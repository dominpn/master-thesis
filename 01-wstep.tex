
\chapter{Wstęp}

\indent Rekomendacja (zgodnie z jedną z definicji zawartej w słowniku języka polskiego) oznacza “pozytywną opinię wydaną o kimś” \cite{pwn}. Każdy z nas w swoim życiu spotkał się z poleceniem przez znajomego ostatnio przeczytanej książki lub obejrzanego filmu. Na tej podstawie można wywnioskować, że pierwsze mechanizmy rekomendacji powstały na długo przed rozpowszechnieniem handlu internetowego, gdzie aktualnie zyskują coraz większą popularność.

\indent W dobie ciągłego postępu oraz globalizacji systemy rekomendacyjne wchodzące w skład złożonych serwisów odgrywają coraz większą rolę. Wcześniej stosowane sposoby prezentacji (takie jak reklama czy newslettery) produktu stały się dla użytkowników zbyt natarczywe lub nie przynosiły firmom oczekiwanego rezultatu. Stąd korporacje i przedsiębiorstwa inwestują coraz większe pieniądze w implementację i wdrożenie systemów rekomendacyjnych.

\indent Głównym elementem pracy jest dokonanie przeglądu mechanizmów rekomendacji, podział ze względu na zastosowanie oraz opracowanie bezpiecznego protokołu oraz porównanie jego wydajności i skuteczności.

\indent Pierwszy rozdział przedstawia ogólną charakterystykę systemów rekomendacyjnych, ich cechy charakterystyczne oraz założenia. Ponadto przedstawiono najczęściej stosowane oraz sprawdzone komercyjnie rozwiązania podczas implementacji tego typu oprogramowania. Także omówiono problemy oraz wyzwania z którymi muszą zmagać się programiści tworząc nowoczesne systemy rekomendacyjne.  

\indent W kolejnym rozdziale został przedstawiony problem związany z bezpieczeństwem danych użytkownika oraz możliwością jego identyfikacji po wystawionych ocenach.

\indent W dalszej części pracy pokazano możliwe rozwiązanie problemu związanego z identyfikacją użytkownika.
W podsumowaniu omówiono otrzymane wyniki, możliwości udoskonalenia mechanizmu bezpieczeństwa 



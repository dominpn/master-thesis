
\chapter{Aspekt bezpieczeństwa w systemach rekomendacyjnych}
Systemy rekomendacyjne muszą gromadzić wiele danych o użytkownikach w celu wykonania trafnego przewidywania potencjalnych obiektów zainteresowania. W zależności od metody serwis dokonuje porównania  profili użytkowników, oblicza podobieństwo pomiędzy użytkownikami czy grupuje użytkowników o zbliżonych cechach. Gromadzone w tym celu dane mogą pozwolić na identyfikacje użytkownika przez co twórcy systemu muszą wybierać pomiędzy prywatnością użytkowników, a skutecznością w dokonywaniu rekomendacji. Taki system z wrażliwymi danymi użytkowników może stać się celem grup hakerskich chcących pozyskać tak cenne informacje. Stąd zapewnienie bezpieczeństwa podczas przesyłania danych pomiędzy serwisami oraz dokonywania obliczeń na nich powinno stać się priorytetem. Jednak większość prac skupia się na zwiększeniu skuteczności czy wydajności \cite{recent_developments}.

\section{Prywatność}

\section{Perturbacja danych}

\section{Bezpieczne obliczanie wieloczęściowe}

\section{Szyfrowanie homomorficzne}